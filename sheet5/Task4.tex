To prove : The stationary distribution of the Markov chain with transition matrix Q is P.\\
Setting up the sample probability space (U,P) such that we have a functional value D =ZP we need to prove that\\
$D(u)q_{uv}$=$D(v)q_{vu4}$. \\
For u=v, the result holds as trivial. \\
If u and v differ in atleast 2 coordinates , $q_{uv}=0 $and the result still holds true.
If u and v differ only by one coordinate, then $q_{uv}$ is given by  $q_{uv}$=  $  1/l . P(v_{i} | u_{1}, u_{2}... u{l})$ \\
$D(u).q_{uv}$= D(u). 1/\emph{l} $\cdot$ $P(v_{i} | u_{1}, u_{2}... u_{l} )$ \\ 
$\implies $  D(u).$q_{uv}$= D(u). 1/\emph{l} . $$(\frac{D(v)}{ \sum_{0 \in D} D(u_{1}, ..0...u_{l})})$$  \\
Likewise, \\
D(v).$q_{vu}$= D(v). 1/l $\cdot$ $$(\frac{D(u)}{\sum_{0 \in D} D(v_{1}, ..0...v_{l})})$$  \\
Also, for every i $\neq$ j ,$ v_{i}= u_{j}$ or $ v_{j}=u_{j}$ \\
We have,\\
$$\sum_{0 \in D} D(u_{1}, ..0...u_{l})$$ =  $$\sum_{0 \in D} D(v_{1}, ..0...v_{l})$$ = S  \\

Thus, D(u).$q_{uv}$= D(u). 1/l. D(v)/S = D(v) .1/l . D(u)/S = D(v).$q_{vu}$.\\
By Lemma ,6.4, for a probability vector r satisfying $r_{i}$ .$q_{ij}$=$r_{j}.q{ji}$ for a transition matrix Q of Markov chain, r is a stationary distribution. Hence, Proved.

