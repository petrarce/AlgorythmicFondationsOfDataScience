To prove : The stationary distribution of the Markov chain with transition matrix Q is P.\\
Setting up the sample probability space (U,P) such that we have a functional value $D(x) =ZP(x)$ we need to prove that
$D(u)q_{uv}$=$D(v)q_{vu}$. \\
For u=v, the result holds as trivial. \\
If u and v differ in atleast 2 coordinates , $q_{uv} = q_{vu}=0$ and the result still holds true.\\
If u and v differ only by one coordinate, then $q_{uv}$ is given by  $q_{uv}$=  $  \dfrac{1}{l} \cdot P(v_{i} | u_{1}, u_{2}... u{l})$ \\
$D(u)\cdot q_{uv} = D(u)\cdot \dfrac{1}{l} \cdot P(v_{i} | u_{1}, u_{2}... u_{l} ) = 
 D(u)\cdot \dfrac{1}{l}\cdot \left(\frac{D(v)}{ \sum_{o \in \mathbb{D}} D(u_{1}, ..., u_{i-1}, o, u_{i+1}, ...,u_{l})}\right)$\\
 Likewise: $D(v)\cdot q_{vu} = D(v) \cdot \dfrac{1}{l} \cdot \dfrac{D(u)}{\sum_{o \in \mathbb{D}} D(v_{1}, .., v_{i-1}, o,v_{i+1}, ...v_{l})})$  \\
Since $\forall j \neq i: v_{j} = u_{j} \text{(where i is a position, in which v and u differ)}\implies$\\
$ \sum_{o \in \mathbb{D}} D(u_{1}, .., u_{i-1}, o, u_{i+1},...u_{l}) =  \sum_{o \in \mathbb{D}} D(v_{1}, ..v_{i-1}, o,v_{i+1},...v_{l}) = S$.\\
From here, $D(u) \cdot q_{uv} = D(u)\cdot  \dfrac{1}{l}\cdot  \dfrac{D(v)}{S} = D(v) \cdot \dfrac{1}{l} \cdot  \dfrac{D(u)}{S} = D(v)\cdot q_{vu}$.\\
Thus by lemma, 6.4 P is a stationary distribution of Markov-Chain, defined by Q.

